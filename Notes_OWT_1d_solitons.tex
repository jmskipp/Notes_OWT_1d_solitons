\documentclass[11pt,a4paper]{article}
\usepackage[latin1]{inputenc}
\usepackage{amsmath}
\usepackage{amsfonts}
\usepackage{amssymb}
\usepackage{mathtools}
\usepackage{graphicx}
\usepackage{bigints}
\usepackage{natbib}
\usepackage{xcolor}
%\usepackage{tikz}


\textwidth 17 cm
\oddsidemargin -0.5 cm
\evensidemargin -0.5 cm
\textheight 21 cm
%\addtolength\topmargin{-20pt}
%\addtolength\textheight{25pt}
%
%%\parindent 0pt
%\parskip 0.5cm

\DeclareMathOperator{\arccosh}{arccosh}
\DeclareMathOperator{\sech}{sech}
\DeclareMathOperator{\Res}{Res}

\title{Notes on the 1D Nonlinear Schr{\"o}dinger Equation, its soliton solution, and its long-wave Optical Wave Turbulence modification}
\author{Jonathan Skipp}
\date{}	
	
\begin{document}


    \maketitle
\begin{abstract}
These notes contain several results relating to the 1D NLSE, its bright soliton solution, and its modification for Optical Wave Turbulence in the long-wave regime.
 \end{abstract}

\section{Derivation the bright soliton solution}

This first section contains a reproduction of the derivation for the single-bright-soliton solution of the 1D Nonlinear Schr{\"o}dinger Equation (NLSE), based on the method outlined in the lecture notes Sergey Nazarenko for the University of Warwick course MA4L0, Spring 2016, pp.46-50. 
	This derivation is carried out for the normalisation of the NLSE with all coefficients equal to unity. I also state how how to convert the equation, and its soliton solution, to  other normalisations. In addition, I state the Fourier transform and waveaction spectrum of the soliton.

Consider the one-dimensional (1D), cubic, focusing, NLSE, written in units in which \textbf{all coefficients are unity}:
\begin{equation}
	\label{eq:NLS}
	i\frac{\partial\psi}{\partial t} + \frac{\partial^2 \psi}{\partial \mathrm{x}^2} + |\psi|^2 \psi =0.
\end{equation}
We seek a bright soliton solution, i.e.\ a localised pulse of the form 
\begin{equation}
	\label{eq:soli_ansatz}
	\psi = e^{i(\beta \mathrm{x}-\gamma t)} f(\xi),
\end{equation}
where the constants $\beta$ and $\gamma$ are real, and $f(\xi)$ is a real function of the travelling coordinate $\xi = \mathrm{x} - ct$. For a localised pulse we have $f(\xi)\to0$ as $\xi\to\pm\infty$. 

We substitute~\eqref{eq:soli_ansatz} into~\eqref{eq:NLS} and set $\beta=c/2$ in order to eliminate the term $v'$. Defining $\mu = (c/2)^2 -\gamma$ we obtain an ODE for the soliton profile
\begin{equation}
	\label{eq:profile_ODE}
	f'' - \mu f + f^3 = 0.
\end{equation}
or
\begin{equation*}
	f'' = -\frac{\mathrm{d}}{\mathrm{d} f}U(f),
\end{equation*}
which is Newton's second law for a particle with dynamical variable (``position'') $f(\xi)$ depending on ``time'' $\xi$, moving in a quartic potential $U(f)=f^4/4-\mu f^2/2$. For such a particle the energy is conserved:
\begin{equation*}
	E = \frac{f'^2}{2} + U(f) = \mathrm{const}.
\end{equation*}
Considering the dynamics of such a particle in a potential well of shape $U(f)$, most trajectories are periodic. Translating back to the solution $\psi(\mathrm{x},t)$, this would represent a periodic nonlinear wave rather than a localised pulse. Such a pulse is given only by the homoclinic orbits of the system~\eqref{eq:profile_ODE} that start and end at $f=0$, $f'=0$, corresponding to $E=0$. We choose the right orbit with positive $f$. 

The choice of trajectory with $E=0$ can also be seen directly by going back to~\eqref{eq:profile_ODE}, multiplying by $f'$, and noting that each term on the LHS can be written as a derivative with respect to $\xi$. Noting that $f$ decays at infinity, we integrate immediately to obtain
\begin{equation}
	\label{eq:E=0}
	\frac{f'^2}{2} + \frac{f^4}{4}-\mu \frac{f^2}{2} = E = 0.
\end{equation}

Equation~\eqref{eq:E=0} is separable; we have
\begin{equation*}
	\int\! \frac{\mathrm{d}f}{\sqrt{\mu f^2 - f^4/2}} = \int\! \mathrm{d}\xi = \xi.
\end{equation*}
Changing variables to $z=\sqrt{2\mu}/f$, the LHS beomes
\begin{equation*}
	\frac{1}{\sqrt{\mu}} \int\! \frac{\mathrm{d} z}{\sqrt{z^2-1}} = \frac{1}{\sqrt{\mu}} \arccosh(z).
\end{equation*}
In terms of $f$, the soliton profile is
\begin{equation*}
	f = \frac{\sqrt{2\mu}}{\cosh(\sqrt{\mu}\xi)}
\end{equation*}
which we substitute, together with $\gamma = (c/2)^2 - \mu$, into~\eqref{eq:soli_ansatz} to find the form of the bright soliton
\begin{equation*}
	\psi(\mathrm{x},t) 
		=  \sqrt{2\mu} \, \sech\!\big[\sqrt{\mu}(\mathrm{x}-ct)\big]  \, 
			e^{i\left(\frac{c}{2}\mathrm{x}+\left[\mu- \left(\frac{c}{2}\right)^2\right]t \right) }.
\end{equation*}
Finally, we note that the NLSE is invariant to spatial translation and to $U(1)$ phase shifts, allowing us to insert constants $\mathrm{x}_0$ and $\phi_0$ to obtain the more general solution
\begin{equation}
	\label{eq:soliton_plus_consts}
	\psi(\mathrm{x},t) = \sqrt{2\mu} \, \sech\!\big[\sqrt{\mu}(\mathrm{x}-\mathrm{x}_0-ct)\big]  \, e^{i\left( \frac{c}{2}(\mathrm{x}-\mathrm{x}_0) + \left[\mu-\left(\frac{c}{2}\right)^2\right] t \right) } \, e^{i\phi_0}.
\end{equation}



\subsection{Conversion to the ``$(1/2)\partial_{xx}$'' notation}
\label{subsec:Gelash_notation}
 
Another common notation for the NLSE has a factor of one-half in front of the kinetic term, namely
\begin{equation}
	\label{eq:NLS_GELASH}
		i\frac{\partial \psi}{\partial t} + \frac{1}{2}\frac{\partial^2 \psi}{\partial x^2} + |\psi|^2 \psi =0.
\end{equation}
Note in equation~\eqref{eq:NLS_GELASH} I have written $x$ in an upright font to distinguish it from the spatial coordinate $\mathrm{x}$ in~\eqref{eq:NLS}. 

The bright soliton solution for~\eqref{eq:NLS_GELASH} is
\begin{equation}
	\label{eq:soliton_plus_consts_GELASH}
	\psi(x ,t) = a \frac{e^{iv(x -x _0-vt) + \frac{1}{2}i(a^2+v^2)t + i\theta}}{\cosh\left[ a(x -x _0-vt) \right]},
\end{equation}
for example, see Gelash and  Agafontsev, Phys.\ Rev.\ E 98, 042210 (2018). (Note carefully the signs of the $v^2t$ factors in the exponentials.) 

To convert between the notation of~\eqref{eq:NLS},~\eqref{eq:soliton_plus_consts} and that of~\eqref{eq:NLS_GELASH},~\eqref{eq:soliton_plus_consts_GELASH}, we simply set
\begin{equation*}
	\mathrm{x} = \sqrt{2} \, x, 	\qquad 
	\mu= \frac{a^2}{2}, 		\qquad
	c=\sqrt{2} \, v,			\qquad
	\phi_0 = \theta.
\end{equation*}



\subsection{Conversion to the NLS with arbitrary coefficients}

More generally we can rescale space and the field amplitude in the NLSE~\eqref{eq:NLS} and its 1-soliton solution~\eqref{eq:soliton_plus_consts} via
\begin{equation*}
	\mathrm{x}     = \frac{1}{\sqrt{C_l}} x , \qquad
	\psi = \sqrt{C_n} \Psi, \qquad
	c     = \frac{v}{\sqrt{C_l}},
\end{equation*}
to obtain the equation
\begin{equation*}
	i\frac{\partial\Psi}{\partial t} + C_l \frac{\partial^2 \Psi}{\partial x ^2} + C_n |\Psi|^2 \Psi =0,
\end{equation*}
which has the soliton solution
\begin{equation*}
	\Psi(x ,t) = 
				A  \,
	 			\sech\!\left[  \kappa (x -x _0-vt)  \right]  \, 
	 			\exp\!\left(i \left\{  \frac{v}{2C_l} (x -x _0)  \,+ \, \left[ \frac{A^2C_n}{2}  -  \frac{1}{C_l} \left( \frac{v}{2}\right)^2 \right]t \right\} \right)\, 
	 			\exp(i\phi_0).
\end{equation*}
%\begin{equation*}
%	\Psi(x ,t) = 
%				\sqrt{\frac{2\mu}{C_n}}  \,
%	 			\sech\!\left[  \sqrt{\frac{\mu}{C_l}} (x -x _0-vt)  \right]  \, 
%	 			\exp\!\left(i \left\{  \frac{v}{2C_l} (x -x _0)  \,+ \, \left[ \mu  -  \frac{1}{C_l} \left( \frac{v}{2}\right)^2 \right]t \right\} \right)\, 
%	 			\exp(i\phi_0).
%\end{equation*}
where the soliton amplitude $A$, the the inverse $\kappa$ of its characteristic width are
\begin{equation*}
	A = \sqrt{\frac{2\mu}{C_n}}, \qquad \mathrm{and} \qquad
	\kappa = A\sqrt{\frac{C_n}{2C_l}},
\end{equation*}
respectively.


\section{Waveaction and spatio-temporal spectra of the 1-soliton solution (``$(1/2)\partial_{xx}$'' notation)}

Reverting to the normalisation of section~\ref{subsec:Gelash_notation}, we consider the 1-soliton solution~\eqref{eq:soliton_plus_consts_GELASH}, setting $x_0$ and $\theta$ set to $0$ for convenience (nonzero values result in an overall phase shift in the Fourier transforms, which make no difference to the spectra):
\begin{equation}
	\label{eq:soliton}
	\psi(x ,t) = a \frac{e^{iv(x-vt) + i(\frac{a^2+v^2}{2})t}}{\cosh\left[ a(x-vt) \right]}.
\end{equation}
In this section we find the spatial Fourier transform and  spatio-temporal Fourier transform of~\eqref{eq:soliton}, and from them obtain the respective waveaction and spatio-temporal spectra. 

To do so we need the result
\begin{equation}
	\label{eq:FTsech}
	J(\kappa ; x)   
	\coloneqq   \int_{-\infty}^\infty    \frac{2e^{i\kappa x}}{e^x+e^{-x}}  \,\mathrm{d}x
	= \pi\sech\left( \frac{\pi \kappa }{2}\right),
\end{equation}
the proof of which is a pleasantly diverting exercise in contour integration, which we give in Appendix~\ref{app:FTsech}.

\subsection{Spatial FT and waveaction spectrum}
The spatial Fourier transform of~\eqref{eq:soliton} is
\begin{align*}
	\hat{\psi}(k,t) &=  \int_{-\infty}^\infty \!  a\frac{e^{iv(x-vt) + i(\frac{a^2+v^2}{2})t}}{\cosh\left[ a(x-vt) \right]} e^{-ikx}  \,\mathrm{d}x   \\
						&=  \int_{-\infty}^\infty \!  \frac{ 2 e^{iv(x-vt)} e^{-ik(x-vt)} }{e^{a(x-vt)}+e^{-a(x-vt)}}   \,\mathrm{d}x  
								\cdot ae^{i(\frac{a^2+v^2}{2})t}e^{-ikvt}									\\
						&= \int_{-\infty}^\infty \!  \frac{ 2 e^{i(v-k)\xi} }{e^{a\xi}+e^{-a\xi}}   \,\mathrm{d}\xi  
								\cdot ae^{i(\frac{a^2+v^2}{2}-kv)t}
\end{align*}
where in the last step we have made a Galilean transform into the frame moving at the soliton speed, with coordinates  $\xi=x-vt$ and $t'=t$ and immediately dropped the prime on the time coordinate (in particular for the Galilean transform $\mathrm{d}x = \mathrm{d}\xi$). Using~\eqref{eq:FTsech} with $(v-k)/a \to \kappa$ and $a\xi\to x$ we then obtain the spatial Fourier transform of the 1-soliton solution
\begin{equation}
	\label{eq:psihat_soli}
	\hat{\psi}(k,t) 	= \pi \sech\left( \frac{\pi(k-v)}{2a} \right) e^{i(\frac{a^2+v^2}{2}-kv)t}		
\end{equation}.
The waveaction spectrum follows:
\begin{equation}
	\label{eq:nk_soli}
	n_k(t) \propto |\hat{\psi}_k|^2 = \pi^2 \sech^2\left( \frac{\pi(k-v)}{2a} \right).
\end{equation}
{\color{red} WHAT ABOUT THE NORMALISATION ?? }.

\subsection{Spatio-temporal transform and spectrum}

The temporal Fourier transform of~\eqref{eq:psihat_soli} is trivial, but note that the sign in the exponent of the transform kernel $e^{i\omega t}$ is positive, as the full spatio-temporal decomposition into plane waves goes (schematically) as $\psi(x,t) \sim \tilde{\psi}(k,\omega) \exp[i (kx - \omega t)]$. The spatio-temporal Fourier coefficient is thus
\begin{align*}
	\tilde{\psi}(k,\omega)  
			&=  \int_{-\infty}^{\infty} \! \hat{\psi}(k,t) e^{i\omega t} \, \mathrm{d}t 				\\
			&=  \int_{-\infty}^{\infty} \!  e^{i(\frac{a^2+v^2}{2}-kv)t}	e^{i\omega t} \, \mathrm{d}t \cdot \pi \sech\left( \frac{\pi(k-v)}{2a} \right).
\end{align*}
Using the identity $\int_\mathbb{R} \exp[i(\omega+\omega_0)t] \mathrm{d}t = 2\pi\delta(\omega+\omega_0)$ we obtain the spatio-temporal Fourier transform
\begin{equation}
	\label{eq:psitilde_soli}
	\tilde{\psi}(k,\omega)  = 2\pi^2 \delta\left(\omega+\frac{a^2+v^2}{2} -kv\right) \sech\left( \frac{\pi(k-v)}{2a} \right)
\end{equation}
and the spatio-temporal spectrum varies as
\begin{equation}
	\label{eq:nkomega_soli}
	n_{k\omega} \propto |\tilde{\psi}|^2 = 4\pi^4 \left[ \delta\left(\omega+\frac{a^2+v^2}{2} -kv\right) \right]^2  \sech^2\left( \frac{\pi(k-v)}{2a} \right).
\end{equation}

In~\eqref{eq:nkomega_soli} the argument of the delta function implies that the spatio-temporal spectrum of the soliton will be zero except on the line
\begin{equation}
	\label{eq:nkomega_soli_line}
	\omega = vk - \frac{a^2+v^2}{2},
\end{equation}
i.e. in the $k$--$\omega$ plane the soliton spectrum will have a gradient of $v$. The dependence of the sech profile with $(k-v)$ implies that in the $k$--$\omega$ plane, the soliton spectrum will be centred horizontally on $k_c=v$. Thus from~\eqref{eq:nkomega_soli_line} the soliton spectrum will be centred vertically on $\omega_c= (v^2-a^2)/2$.

\appendix

\section{Proof of eq~\eqref{eq:FTsech}}
\label{app:FTsech}

Consider the contour integral
\begin{equation*}
	\underbrace{\oint_C    		\frac{2e^{i\kappa z}}{e^z+e^{-z}}  \,\mathrm{d}z}_{I_C} =
	\underbrace{\int_{-R}^R     \frac{2e^{i\kappa x}}{e^x+e^{-x}}  \,\mathrm{d}x}_{I_R} +
	\underbrace{\int_\Gamma  \frac{2e^{i\kappa z}}{e^z+e^{-z}}  \,\mathrm{d}z}_{I_\Gamma}
\end{equation*}
where $C$ is the contour formed by the length along the real axis $-R \leq x \leq R$ and a semicircular closure $\Gamma$ of radius $R$. If $\kappa>0$ (respectively if $\kappa<0$) we take $\Gamma$ to lie in the upper (lower) half-plane, so that in the limit $R\to\infty$ we have $I_\Gamma\to0$; also
$I_R\to J(\kappa;x)$, and therefore
\begin{equation*}
	J(\kappa ; x) = \lim_{R\to\infty} I_C.
\end{equation*}

The integrand in $I_C$ has simple poles at $z_n = i\pi(n+1/2)$ with $n\in\{0,1,2,\ldots\}$. Defining $\delta z = z-z_n$, the $n$-th residue is
\begin{align*}
\Res_n  &= \lim_{\delta z\to 0} \delta z \, \frac{2e^{i\kappa (z_n+\delta z)}}{e^{z_n + \delta z}+e^{-z_n -\delta z}}						\\
			&=  \lim_{\delta z\to 0} \delta z \, 
					\frac{  2e^{-\pi\kappa (n+1/2)}  }{  \underbrace{e^{i\pi n}}_{(-1)^n} \underbrace{e^{i\pi/2}}_i e^{\delta z}
							+\underbrace{e^{-i\pi n}}_{(-1)^n}\underbrace{e^{-i\pi/2}}_{-i} e^{-\delta z}  } 									\\
			&= \lim_{\delta z\to 0} \delta z \, \frac{  2(-i)(-1)^n e^{-\pi\kappa (n+1/2)}  }{  e^{\delta z} - e^{-\delta z}  }						\\
			&= \lim_{\delta z\to 0} \delta z \, 
					\frac{  2(-i)(-1)^n  e^{-\pi\kappa (n+1/2)}  }{  [1+\delta z + \mathcal{O}(\delta z^2)] - [1-\delta z+\mathcal{O}(\delta z^2)]  }.  \\
			&= \lim_{\delta z\to 0} \delta z \, 
					\frac{  2(-i)(-1)^n  e^{-\pi\kappa (n+1/2)}  }  {  2 \, \delta z + \mathcal{O}(\delta z^2)  }.  											\\
			&=   (-i)(-1)^n e^{-\pi\kappa (n+1/2)}  .
\end{align*}
Thus by the residue theorem the $n$-th contribution to $I_C$ due to $\Res_n$ is
\begin{equation*}
	(2\pi i) \Res_n = (2\pi i) \left[  (-i)(-1)^n e^{-\pi\kappa (n+1/2)}  \right] =  2\pi e^{-\pi\kappa /2} \left(-e^{-\pi\kappa } \right)^n.
\end{equation*}
Taking the $R\to\infty$ limit we sum an infinite number of these contributions, which is easy to evaluate as the sum is geometric:
\begin{align*}
\lim_{R\to\infty}I_C &= \sum_{n=0}^\infty  2\pi e^{-\pi\kappa/2} \left(-e^{-\pi i\kappa } \right)^n 		\\
	 						 &= 2\pi e^{-\pi\kappa/2} \frac{1}{1+e^{-\pi\kappa}}										\\
						  	 &= \frac{2\pi}{ e^{\pi \kappa/2} + e^{-\pi \kappa/2}}.
\end{align*} 
Equating this to $J(\kappa ; x)$ we finally we obtain~\eqref{eq:FTsech}:
\begin{equation*}
J(\kappa ; x) = \pi \sech\left( \frac{\pi \kappa}{2} \right).
\end{equation*}

\end{document}
